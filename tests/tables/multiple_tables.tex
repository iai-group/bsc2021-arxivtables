qokwjnmeiopwqjkeiopqjkwiopepowqkop[e
qwe
wqew
qe
wqe
qwwqe



\begin{table}[t]
	\caption{Mapping the action set used in this paper to high-level QRFA categories.}
	\captionshrink
	\small
	\begin{tabular}{@{~}lp{6.7cm}}
	\toprule
	\textbf{Category} & \textbf{Actions} \\
	\midrule
	Query & Reveal, Disclose, Non-disclose, Revise, Rene, Expand, Inquire, List, Compare, Subset, Similar, Navigate, Repeat, Interrupt, Interrogate \\
	Request & Inquire, Elicit, Clarify, Suggest \\
	Feedback & Back, More, Note, Complete \\
	Answer & Show, List, Similar, Subset, Repeat, Back, More, Record, End \\
	\bottomrule
	\end{tabular}
\label{tbl:actions_qrfa}
\end{table}

\textbf{asdujjhasioudjoidjiowqjeiojqwoijeioqjwoiejioqwjioeoiqwjioewq}

\begin{table}[t]
\footnotesize
  \centering
  \caption{Most frequent patterns based of POS-tagger.
  NN is noun singular, IN is preposition/subbordinating conjunction, CD is cardinal digit, JJ is adjective, NNP is proper noun, NNS is noun plural, and VBN is past participle taken.  (acc=95\%, errors lie at e.g. Greek (both adj and noun))}
  % https://medium.com/@gianpaul.r/tokenization-and-parts-of-speech-pos-tagging-in-pythons-nltk-library-2d30f70af13b
  \begin{tabular}{lll}
    \toprule
    Pattern & Occurrence & Example \\
    \midrule
    NNS IN NNP & 47167 & \emph{Writers from Toronto} \\
    CD NNS IN NNP & 33858 & \emph{1929 establishments in California} \\
    JJ NNS & 31482 & \emph{Spanish films} \\
%    NNP NNP NNS & 31294 & ``Ancient Greek mathematicians'' \\
    CD IN NNP & 27484 & \emph{1959 in Antarctica} \\
    NN & 23434 & \emph{Agriculture} \\
    NNS IN NNP NNP & 22996 & \emph{Compositions by George Gershwin}\\
    NNP NNS & 19262 & \emph{Vowel letters} \\
    JJ NNS IN NNP & 18615 & \emph{Duty-free zones of Europe} \\
    NNP NNP & 17558 & \emph{Nobel Prize} \\
    NNS VBN IN NNP NNP & 15286 & \emph{Organizations based in Los Angeles} \\
    \bottomrule
  \end{tabular}
  \label{tbl:pattern}
\end{table}
