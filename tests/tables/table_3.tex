\documentclass{article}
\usepackage[utf8]{inputenc}
\begin{document}
\begin{table}[t]
    \begin{center}
        \begin{tabular}{l|c|c}
            & $p$ is in $\str(y_1)$                                                    & $p$ is in $\str(y_2)$ \\ \hline%\hline
            $r(y_1) \leq \ell(y_2)$ & $\begin{array}{c}(+1,0) \to (+1,0) \\ (0,-1) \to (+1,0)\end{array}$   & $\begin{array}{c}(+1,0) \to (+1,0) \\ (0,-1) \to (0,-1)\end{array}$ \\ 
            \hline
            $r(y_1) > \ell(y_2)$ & $\begin{array}{c}(+1,0) \to (+1,0) \\ (0,-1) \to (0,-1)\end{array}$   & $\begin{array}{c}(+1,0) \to (0,-1) \\ (0,-1) \to (0,-1)\end{array}$ \\
        \end{tabular}
    \end{center}
    \caption{
        The effect on $x$ after a closing parenthesis was inserted at position $p$.
        The effects depend on the effect on the children $y_1$ and $y_2$ of $x$:
        for example, an entry '$(0,-1) \to (+1,0)$' in the column '$p$ is in $\str(y_1)$' means that
        if the change operation has effect $(0,-1)$ on $y_1$ then the change operation has effect $(+1,0)$ on $x$.
    }
    \label{tbl:table-d1-cases}
  \end{table}
\end{document}